title = Welcome
subtitle = Discovery awaits!
metadesc = This is a demo meta description entry
metakeys = demo,demo keys
tags = freedom
layout = homepage.html
###

Welcome to your new installation of ergo-cms.


h2. Getting Oriented

# Edit @config.ergo.js@ to change configuration and basic set up parameters, such as site name, site url and social media usernames.
# To help you get started, we've added a couple of pages to this skeleton file, which are located in the @source@ folder. Feel free to delete/rename them them as you see fit. 
Anything in the @source@ folder will be processed and copied to the output folder. For example, this file, @index.tex@ will become @index.html@ in the @output@ folder once it is built.
# If you want to change the menu at the top, please edit @_partials/menu.tex@. You can also change @_partials/footer_text.tex@ to change the footer too.
# If you don't want to use textile, but prefer Markdown: simply change the extensions of these files to @.md@ after reformatting according to markdown. Nothing else needs to be done. 


h2. Quick Commands

ergo has a simple command set. If you ever wish to explore, just type @ergo --help@. If you want more info on a particular command (eg 'view') then type @ergo view --help@. If you didn't install ergo globally, then you can type in @npm run ergo -- --help@ or @npm run ergo -- view --help@ instead.

h3. ergo build

When you're ready you can call @ergo build@ from _anywhere_ inside the folder structure of the website and the resulting static website will be created in the @output@ folder.


h3. ergo view

@ergo view@ allows you to browse your website from a local web server.

If you will be making changes to the @source@ folder, then you might want to run @ergo view --watch@ instead. The @--watch@ parameter causes ergo to rebuild your website if you make changes, allowing you to see them in the browser, quickly! Note also that @-w@ is shorthand for @--watch@, so you can actually type @ergo view -w@.

h2. Changing themes

If you've already downloaded a theme, you can change to it by editing your @config.ergo.js@ file. 

Otherwise, you can download and install most themes easily with @ergo theme install theme-name@. Please note, that you may need to make some changes to your settings and files to get the theme to work. The theme will have a README file of it's own which can guide you in the steps to take. 

When you are ready, run @ergo build --clean@ to make the switch official!

{
title:"Welcome",
metadesc: "This is a demo meta description entry",
metakeys: "demo,demo keys",
tags:"freedom"
}


If you are reading this for the first time: _Welcome to ergo-cms!_ Your first step to understanding where to get started will be to browse these files in a explorer/finder window. But first, you may want to read this in a browser (if you aren't already).

If you are not yet viewing this page in a browser window, try running:

bc. ergo build && ergo view

If you didn't install ergo globally, that's ok. Simply, run this instead:

bc. npm install && npm run build && npm run view

If everything went well, you can open your browser at "http://localhost:8181":http://localhost:8181/, to be taken to the 'live' version of this page. If however, you ran into problems, then skip down to the "Installation":#install section, below.

Then, when you're ready for more, go the "first blog post":/blog/a-first-post.html.

You may also care to check out the "bootstrap":http://getbootstrap.com/ version of these files, to show how adaptable this all is. Simply download or clone from "ergo-cms/ergo-skel-bootstrap":https://github.com/ergo-cms/ergo-skel-bootstrap and then @ergo build && ergo view@.

If you're planning to do some editing of your files (whether it's writing new articles, or editing a stylesheets, or javascript), you might want to enable _'Watch Mode'_:

bc. ergo build && ergo view --watch

Ergo will now automatically rebuild when changes in the source folder are detected. (If you're running the _local_ version, then type @npm run build && npm run view -- --watch@. Note the extra @--@!)


h2(#install). Installation

If you've simply cloned or unzipped this skeleton file from "github":https://github.com/ergo-cms, don't worry, we can get you up and running. You have a choice to make unfortunately: _local_ or _global_ installation?

A _local_ installation is a simple choice, as it is very simple to do. Simply type:

bc. npm install

...from within the skeleton directory. From here on, if you ever see the command @ergo [command] [options]@, you can simply replace it with @npm run [command] -- [options]@ to achieve similar results. Note that the @--@ is important (but you'll quickly notice it if you forget!)

A _global_ installation takes up exactly the same amount of room on your hard drive. 

bc. npm install ergo-cli -g

In some instances, you may need to run the command as the 'root' user: @sudo npm install ergo-cli -g@

The benefit of a _global_ installation is mainly if you are authoring more than one website. You don't need to call @npm install@ inside each skeleton folder before using it -- it's ready immediately after unzipping/cloning it.


h2. Quick Commands

ergo has a simple command set. If you ever wish to explore, just type @ergo --help@. If you want more info on a particular command (eg 'view') then type @ergo view --help@.

h3. ergo build

When you're ready you can call @ergo build@ from _anywhere_ inside the folder structure of the website. 


h3. ergo view

@ergo view@ allows you to browse your website from a local web server.

If you will be making changes to the @source@ folder, then you might want to run @ergo view --watch@ instead. The @--watch@ parameter causes ergo to rebuild your website if you make changes, allowing you to see them in the browser, quickly! Note also that @-w@ is shorthand for @--watch@, so you can actually type @ergo view -w@.


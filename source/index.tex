{
title:"Welcome",
metadesc: "This is a demo meta description entry",
metakeys: "demo,demo keys",
tags:"freedom"
}


If you are reading this for the first time: _Welcome to ergo-cms!_ Your first step to understanding where to get started will be to browse these files in a explorer/finder window. But first, you may want to read this in a browser (if you aren't already).

If you are not yet viewing this page in a browser window, try running:

bc. ergo view

But, if you didn't install ergo globally, that's ok. Try running this instead:

bc. npm run view

If everything went well, you can open your browser at "http://localhost:8181":http://localhost:8181/, to be taken to the 'live' version of this page. If however, you ran into problems, then skip down to the "Installation":#install section, below

Then, when you're ready to launch, go the "first blog post":/blog/a-first-post.html.

You may also care to check out the "bootstrap":http://getbootstrap.com/ version of these files, to show how adaptable this all is. Simply download or clone from "ergo-cms/ergo-skel-bootstrap":https://github.com/ergo-cms/ergo-skel-bootstrap and then 'npm run view' as before.

h2(#install). Installation

If you've simply cloned or unzipped this skeleton file from "github":https://github.com/ergo-cms, don't worry, we can get you up and running. You have a choice to make unfortunately: _local_ or _global_ installation?

A _local_ installation is a simple choice, as it is very simple to do. Simply type:

bc. npm install

...from within the skeleton directory. From here on, if you ever see the command @ergo [command] [options]@, you can simply replace it with @npm run [command] -- [options]@ to acheive similar results. Note that the @--@ is important (but you'll quickly notice it if you forget!)

A _global_ installation takes up exactly the same amount of room on your hard drive. 

bc. npm install ergo-cli -g

In some instances, you may need to run the command as 'root':

bc. sudo npm install ergo-cli -g

The benefit of a _global_ installation is mainly if you are authoring more than one website. You don't need to call @npm install@ inside each skeleton folder before using it -- it's ready immediately after unzipping/cloning it.

You can also create new websites easily by typing:

bc. ergo init [folder]

And you will get this skeleton website. Use @ergo init --help@ to get a list of all the options in making a new website.

Of course, when you're ready you can call @ergo build@ from _anywhere_ inside the folder structure of the website and if you want to view it, then type @ergo view --watch@ to start a mini-webserver. The @--watch@ parameter causes ergo to rebuild your website if you make changes, allowing you to see them in the browser, quickly!





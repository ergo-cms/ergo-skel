title = Getting Started
subtitle = Introduces some basic commands for ergo-cms
date = 1 Jan 2017
tags = blogging, information
image = post1.jpg
featured = 1
popular = 1
###

This page is a quick intro into getting started with ergo-cms. It is recommended to read the articles on the "ergo-cms":https://ergo-cms.github.io website for more information.

h2. Getting Oriented

# Edit @config.ergo.js@ to change configuration and basic set up parameters, such as site name, site url and social media usernames.
# To help you get started, we've added a couple of pages to this skeleton file, which are located in the @source@ folder. Feel free to delete/rename them them as you see fit. 
Anything in the @source@ folder will be processed and copied to the output folder. For example, this file, @blog/getting started.tex@ will become @blog/getting-started.html@ in the @output@ folder once it is built.
# If you want to change the menu at the top, please edit @_partials/menu.tex@. You can also change @_partials/footer_text.html@ to change the footer too.
# If you don't want to use Textile, but prefer Markdown: simply change the extensions of these files to @.md@ after reformatting according to markdown. Nothing else needs to be done. 

h3. Textile

Textile is the default language of choice in ergo cms. We feel it's better suited to html authoring than Markdown. You can get more information about the syntax here:

* "https://txstyle.org/":https://txstyle.org/
* "http://redcloth.org/hobix.com/textile/":http://redcloth.org/hobix.com/textile/

The package that ergo-cms uses is by "@borgar@":https://github.com/borgar/textile-js and is compatible with the RedCloth version of textile.

Save your files as @.tex@ or @.textile@ to use textile formatting. 

h3. Markdown

Markdown is very popular for developers and is fully supported here by using the "@marked@ package":https://github.com/chjj/marked.

h2. Quick Commands

ergo has a simple command set. If you ever wish to explore, just type @ergo --help@. If you want more info on a particular command (eg 'view') then type @ergo view --help@. If you didn't install ergo globally, then you can type in @npm run ergo -- --help@ or @npm run ergo -- view --help@ instead.

h3. ergo build

When you're ready you can call @ergo build@ from _anywhere_ inside the folder structure of the website and the resulting static website will be created in the @output@ folder. If you want to 'clean' the output folder, for doing a rebuild, use @ergo build --clean@


h3. ergo view

@ergo view@ allows you to browse your website from a local web server.

If you will be making changes to the @source@ folder, then you might want to run @ergo view --watch@ instead. The @--watch@ parameter causes ergo to rebuild your website if you make changes, allowing you to see them in the browser, quickly! Note also that @-w@ is shorthand for @--watch@, so you can actually type @ergo view -w@. 

You can also build at the same time as starting up with: @ergo view -b -w@

h2. Changing themes

If you've already downloaded a theme, you can change to it by editing your @config.ergo.js@ file. 

Otherwise, you can download and install most themes easily with @ergo theme install theme-name@. Please note, that you may need to make some changes to your settings and files to get the theme to work. The theme will have a README file of it's own which can guide you in the steps to take. 

More themes are appearing all the time. Check out "ergo-cms themes":https://ergo-cms.github.io/themes for the latest official list.

When you are ready, run @ergo build --clean@ to make the switch official!

h2. Plugins

Ergo-cms also comes with a range of helpful plugins, and more are being added constantly. Check out "ergo-cms plugins":https://ergo-cms.github.io/plugins for the latest official list.

h2. Deploying

Please check the "Deployment Guide":https://ergo-cms/github.io/articles/deploy.html for latest information on deployment to your webserver.

